%> ------------------------------------------------------------------------------
%> This file contains most of the definitions you will need to put together
%> your thesis. All requirements were taken from the sources below:
%> http://www.cit.cmu.edu/current_students/graduates/thesis_dissertation_policies.html
%> https://www.epp.cmu.edu/graduate/thesis_format_guide.html
%> ------------------------------------------------------------------------------

%> Your title goes here.
\title{Variational Methods for Energy Systems}

%> Your own name goes here.
\author{Henning Lange}
\affiliation{Advanced Infrastructure Systems}

%> Put your previous degrees here.
\bsdegree{Cognitive Science, Universit\"at Osnabr\"uck}
\msdegree{Machine Learning \& Data Mining, Aalto University}

%> Put the month you will be graduating here, NOT the month in which you
%> actually finished the thesis. The only admissible months here are May,
%> August and December.
\Month{May}

%> Year in which you graduated (or plan to graduate).
\Year{2019}

%> Copyright notice. This may be whatever you want. My personal choice is to
%> make it as widely accessible as possible :)
\permission{\textit{Some rights reserved.} Except where indicated, this work
is licensed under a Creative Commons Attribution 3.0 United States License. 
Please see \smallurl{http://creativecommons.org/licenses/by/3.0/us/} for
details.}

%> Keywords.
\keywords{variational inference, sustainaible energy, machine learning,
non-intrusive load monitoring, alternating current optimal power flow.}
\maketitle

%> ------------------------------------------------------------------------------
%> Abstract. Mandatory and very important. Keep it under 350 words.
%> ------------------------------------------------------------------------------
\begin{abstract}
In this work, two engineering problems that could potentially lead to reductions in energy consumption and waste are identified and solutions are proposed that make use of an approximate statistical inference technique called Variational Inference (VI). VI turns statistical inference into an optimization problem by introducing an auxiliary distribution and minimizing a divergence measure between the auxiliary and true distribution. Because of recent successes of Neural Networks for non-linear optimization, modern VI approaches parameterize the auxiliary distribution by Neural Networks which entails that, after training, inference can be extremely fast as it only requires a forward pass through a Neural Network. The engineering problems at hand encompass a sensing problem on the demand-side, namely Non-Intrusive Load Monitoring, as well as a control problem on the generation-side known as Alternating Current Optimal Power Flow. For these problems, VI based algorithm are introduced that allow to quickly obtain approximate solutions to otherwise NP-hard problems.


%We introduce algorithms for the problems at hand that build on top of Variational Inference that dramatically reduce the computational time required to obtain approximate solutions to otherwise NP-hard problems. %by making use of an auxiliary distribution parameterized by Neural Networks that avoids any independence assumption whilst still be computationally efficient.
\end{abstract}

%> ------------------------------------------------------------------------------
%> Dedication. Optional. This is whatever you want.
%> ------------------------------------------------------------------------------
\begin{dedication}
To the one boy, with only one eye.
\end{dedication}


%> ------------------------------------------------------------------------------
%> Acknowledgements. Mandatory. At the very least you should acknowledge
%> your committee and your funding sources.
%> ------------------------------------------------------------------------------
\begin{acknowledgments}
This is a placeholder for acknowledgements.

\end{acknowledgments}